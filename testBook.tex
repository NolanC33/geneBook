\documentclass{book}
\begin{document}
\title{ello m8}
\author{Nolan Crane}

\maketitle
\tableofcontents

\chapter{Intro}


\section{hello}


Two Indexes have been prepared for descendants of each of the 
four progenitors. One giving the Christian names of Cranes, the other, 
names of persons who have intermarried with them ; alphabetically 
arranged. 

Also an index of Christian names of Cranes found in the Addenda. 
In looking for a certain name, unless you know to which line the 
person belongs it will be necessary to examine each of the Ave indexes. 

The number after the name is the consecutive number. Turn to this 
number in the body of the book and you will find the person's family 
record. If the person had no family the number will refer to the birth 
under the parent's name. If the Index does not give the name sought, 
And n;ime of the parent, and look through the list of the children. 
Some names came too late for numbering. Where there are several 
names alike, the year of birth placed before the name may help to 
indicate the one sought. 

The bracketed [ — ] number after the name of a parent refers to the 
number where the person appears as a child. After the name of a parent the pedigree is indicated in parentheses ( ) with small figures above, showing to what generation the name helongs, and giving the names back as far as the progenitor of the line. 

The following abbreviations have been used: b. for born, m. for 
married, unm. for unmarried, (s. settled), d. for died. 





VOLUME I. of the Crane Family Genealogy, published in 1895, 
contains an article on Origin of the name Crane; copies of 
five different armorials belonging to the family ; a sketch of the 
Crane family in England, with pedigrees covering thirteen genera- 
tions, and a brief reference to the first of the name who came to 
New England, together with Henry Crane of Wethersfield and 
Guilford. Conn., and his descendants of nine generations; a list 
of Cranes who served in Connecticut as lawmakers and public 
officials ; also a list of members of the family who served in the 
French and Indian, as well as the Revolutionary war, from 
that State. There has been considerable delay in issuing the 
second volume, and no doubt some interested persons have 
become impatient to see the work in print. But the task 
of tracing the lines from live different progenitors has been 
no easy one. Special care has been exercised in keeping each 
line of descendants independent of the other, although they 
have in some instances intermarried, and in many cases resided 
for years in close proximity to each other. Volume II. has 
exceeded in size our most sanguine expectations, for the 
reason that the descendants of Jasper and Stephen Crane of 
New Jersey have so willingly assisted in supplying records 
from private sources that could in no other way be furnished. 
The early public records of New Jersey, so far as the writer's 
experience extends, furnish a very unsatisfactory field for the 
genealogist. During the early settlements there seems to have 
been no fixed system for recording births, marriages and deaths 
for preservation. Probate Records have supplied the greater 
portion of the early records for the descendants of Jasper and 
Stephen Crane. Some of the early church records, from which 
much was expected, were found to have been destroyed by fire. 
The lack of dates has caused a vast amount of trouble in estab- 
lishing the identity of certain members of the family. "Where 



there seemed any doubt of the identification it has been ex- 
pressed in the context or by a foot-note. 

Although the writer has given a vast amount of time, and a 
considerable sum from his means in prosecuting this work, still 
he is fully aware how difficult is the task of collecting material 
and compiling a work of this nature, and presenting it free from 
inaccuracies. Every effort has been made to avoid errors, and 
wherever different dates or conflicting statements have been 
received regarding the same event, which has frequently been 
the case, careful investigation has followed in order to determine 
if possible the correct statement to be used in the book. To 
the many friends who have in any way contributed to the 
encouragement of this work the writer would here express his 
profound gratitude. Among the names of those who have 
given special aid in its prosecution may be mentioned : Rev. 
Elias Nettleton Crane. Rev. Oliver Crane and James Eells 
Crane, all deceased; the latter died Nov. 19, 1893, in Phila- 
delphia, Pa. ; William M. Crane, Greenville, Mich. ; Stephen S. 
Crane, Maple Hill Farm, High Ridge, Conn. ; Harrison Horton 
Crane, Middletown. N. Y. ; Augustus S. Crane, Elizabeth, N. J. ; 
Miss A. J. Reed, Carmel, N. Y. ; Dorothy N. Law, Dixon, 111. ; 
Anna Russell Vance, Milwaukee, Wis. ; Henry Harmon Noble, 
Albany, N. Y. ; Horatio Grain, Key West, Fla. : and J. M. 
Crane, Kingwood, West Va. 

Records of several families needing further proof to define 
their position in the body of the book have been placed in the 
Addenda with the hope that some of the descendants may be 
able to find the connecting link. In some instances the peculiar 
spelling of names has been retained, believing members of the 
families would prefer to have them presented in that way. 

ElXERY BlCKN'ELL CRAXE, 

Worcester, Mass. 
January, 1900. 


BENJAMIN CRANE. 



Windsor was probably the first town settled in Connecticut by 
the English, and Wethersfield next. The former made its 
beginuing in the year 1633 and the latter in 1634. The people 
who planted these towns were almost without exception from 
Massachusetts. In the year 1621 and for many years thereafter 
practically all settlers bound for New England lauded in the 
colony of New Plymouth or Massachusetts, and emigrated from 
there to the various settlements of their choice. For several 
years Windsor, Wethersfield and Hartford proved the chief 
attractions for settlers locating west of the Connecticut River, 
although a few planted themselves at Saybrook. Many of those 
who early settled at Wethersfield came from Watertown, Mass., 
while the towns of Cambridge and Dorchester furnished a con- 
siderable number of the families for Windsor, and Newtown 
furnished some for Hartford. The emigrants from Dorchester, 
Mass., named their town Dorchester, now Windsor. Wethersfield 
was called Watertown, and Hartford was called Newtown. But 
at the meeting of the General Court of the Connecticut colony 
in 1637 the present name was decided upon and adopted. This, 
however, was not the first session of the General Court of 
this colony Their first session was held April 26, 1636. 

Rev. Henry Smith was the first settled minister in Wethersfield ; 
came there about the year 1636, but was not installed until the 
year 1640 or 1644. and died 1648. 

The early records of Wethersfield are exceedingly interesting 
and voluminous, yet lack system and completeness. The records 
of births, with some deaths, are furnished from 1635 to about 
1666 ; and after an interval of some years the record of deaths 
is again taken up with the year 1670, and the births and mar- 
riages about the year 1692. Subsequent to the year 1700 there 
seems to have been a more complete system of entries through- 
out all the departments. On the whole, however, the records at 
W T ethersfield are perhaps in as good condition in all respects and 


\end{document}